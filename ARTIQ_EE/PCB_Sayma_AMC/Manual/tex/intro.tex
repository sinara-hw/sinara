%\section{Introduction to Micro TCA}




%The MTCA platform is available on the market for over
%ten years. It evolved from telecommunication ATCA standard. The MTCA sandard utilizes ATCA-defined AMC boards
%used directly in dedicated chassis. It also defines MTCA
%Carrier Hub (MCH) which controlls multiple slave boards, known as Advanced Mezzanine Cards (AMCs) via a high-speed digital backplane. AMC card can be equipped with FPGA Mzzanine Cards (FMCs) which are I/O modules pluggable to High-pin Count (HPC) or Low-pin Count (LPC) connector. 
%
%%which consists of Ethernet hub and crate
%%management system. 
%The MTCA crates are available in several
%form-factors for industrial, aviation and military use.
%User can easily extend and adopt the standard to particular application by selection of
%proper chassis, cooling method, computing and connectivity
%technology while keeping same mechanical, electrical standard and software architecture.\\


%MicroTCA (uTCA) is Sinara's preferred form-factor for hardware with high-speed data converters requiring deterministic phase control, such as the Sayma 2.4GSPS smart arbitrary waveform generator (SAWG).
%
%uTCA is a modular, open standard originally developed by the telecommunications industry. It allows a single rack master -- the Micro TCA Carrier Hub (MCH) -- to control multiple slave boards, known as Advanced Mezzanine Cards (AMCs) via a high-speed digital backplane. uTCA chassis and backplanes are available commercially of the shelf (COTS).
%%
%We make use of the most recent extension to the uTCA standard, uTCA.4. Originating in the high-energy and particle physics (HEPP) community, uTCA.4 introduces rear-transition modules (RTMs) along with a second backplane for low-noise RF signals (RFBP). Each RTM connects to an AMC (one RTM per AMC). Typically, the AMCs hold FPGAs and other high-speed digital hardware, communicating with the MCH via gigabit serial links over the AMC backplane. The RTMs hold data converters and other low-noise analog components, controlled by the corresponding AMC. The RFBP provides low-noise clocks and local oscillators (LOs). The RTMs and RFBP are screened from the AMCs to minimise interference from the high-speed digital logic

\section{Glossary}

\begin{description}
	\item[AMC Module or Modul] An AMC Module is a mezzanine or modular add-on card that extends the
	functionality of a Carrier Board. The term is also used to generically refer to the
	different varieties of Multi-Width and Multi-Height Modules.
	\item[Fat Pipes] Ports 4 though 11 of the AMC Connector constitute the Fat Pipes Region. This
	Region of Ports is intended for the assignment of multiple Lane interfaces, also
	called “fat pipes”. Fat Pipe 1 [Ports 4-7], Fat Pipe [Ports 8-11].
	\item[FMC] FPGA Mezzanine Card
	\item[Hot Swap] To remove a component (e.g., an AMC Module) from a system (e.g., an AMC Carrier
	AdvancedTCA Board) and plug in a new one while the power is still on and the
	system is still operating.
	\item[Management Power or MP] The 3.3V power for a Module's Management function, individually provided to each Slot by the Carrier
	\item[IPMB] ntelligent Platform Management Bus. The lowest level hardware management bus
	as described in the Intelligent Platform Management Bus Communications Protocol
	Specification.
	\item[MGT] Multi-Gigabit Transceiver
	\item[MMC] Module Management Controller. The MMC is the required intelligent controller that
	manages the Module and is interfaced to the Carrier via IPMB-Local
	\item[RTM] Rear Transition Module
\end{description}


