\section{Housekeeping Signals}
Both MMC and FPGA can acces to any of I2C buses as is in Figure \ref{I2C}. MMC collects all data from all sensors connected to I2C bus. Then the data can be transfered via IPMI to MCH. For now MCH get only information about AMC and RTM to allow power supply.

\subsection{sensors}
Temperature:\\
%\begin{itemize}
%	\item IC8 (0x4B) -NOR Flash
%	\item IC34 (0x49) -FPGA
%	\item IC35 (0x4A)-under SFPs
%	\item IC36 (0x4F)-power section
%	\item IC37 (0x24) -middle od the board
%\end{itemize}
\begin{longtable}{|c|c|c|c|c|}\hline
	No & Addr. & placement & Type & Accuracy \\ \hline
	IC8 & 0x4B & NOR Flash & LM75 & +/- 2 \\ \hline
	IC34 & 0x49 & FPGA & LM75 & +/- 2  \\ \hline
	IC35 & 0x4A & Under SFPs & LM75 & +/- 2  \\ \hline	
	IC36 & 0x4F & power section & LM75 & +/- 2  \\ \hline
	IC37 & 0x24 & middle of the board & MAX664A & +/- 1 \\ \hline
\end{longtable}

All temperature sensors are tied tohether to one I2C bus - SENS\_I2C. \\


Current:
%\begin{itemize}
%	\item IC27 (0x40)-RTM\_P12V0
%	\item IC28 (0x41)-FMC\_P12V0
%\end{itemize}
\begin{longtable}{|c|c|c|c|c|}\hline
		No & Addr. & placement & Type & Accuracy \\ \hline
		IC27 & 0x40 &RTM\_P12V0& INA219 & +/- 0.2\% \\ \hline
		IC28 & 0x41 &FMC\_P12V0& INA219 & +/- 0.2\% \\ \hline
\end{longtable}


All current sensors are tied tohether to one I2C bus - PM\_I2C. \\

\subsection{Safety interlocks}


Temperature interlock is available on RTM board only and gets activated after reaching 80 degrees. This is hardware interlock and cannot be deactivated. Dedicated LED (LD15) gets on and RTM power supply is off until the temperature is exceeded