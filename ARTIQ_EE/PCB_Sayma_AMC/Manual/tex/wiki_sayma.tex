\section{Sayma}

Sayma is a hardware that supports M-Lab's Smart Arbitrary Waveform Generator(\href{http://m-labs.hk/artiq/manual-master/core_drivers_reference.html?highlight=sawg#module-artiq.coredevice.sawg}{SAWG}) gateware.
Provide 8 channels of
1.2 GSPS 16-bit DACs (2.4 GHz DAC clock) and 125 MSPS 16-bit ADCs. It
consists of an AMC, providing the high-speed digital logic, and a RTM,
holding the data converters and analog components.

The design files are located in
\href{https://github.com/m-labs/sinara/tree/master/ARTIQ_EE/PCB_Sayma_AMC}{ARTIQ\_EE/PCB\_Sayma\_AMC}
and
\href{https://github.com/m-labs/sinara/tree/master/ARTIQ_EE/PCB_Sayma_RTM}{ARTIQ\_EE/PCB\_Sayma\_RTM}
and, the AMC schematic is
\href{https://github.com/m-labs/sinara/blob/master/ARTIQ_EE/Sayma_AMC.pdf}{here}
and the RTM schematic is
\href{https://github.com/m-labs/sinara/blob/master/ARTIQ_EE/Sayma_RTM.pdf}{here}.
The PCBs are double width, mid height AMC module. Sayma AMC

\subsection{Features}\label{features}

\begin{itemize}

\item
  May be used in a uTCA rack or stand-alone operation with fibre-based
  DRTIO link
\item
  Analog input and output front-ends provided by plug-in
  \href{https://github.com/m-labs/sinara/wiki/SaymaAFE}{AFE} modules (eg BaseMod) for maximum
  flexibility.
\item
  Extremely flexible \href{https://github.com/m-labs/sinara/wiki/SinaraClocking}{clocking options}
\item
  Flexible feedback to SAWG parameters planned. Specification
  \href{https://github.com/m-labs/sinara/wiki/Servo}{here}.
\end{itemize}

\subsection{Key AMC Components}\label{key-amc-components}


\noindent

\textbf{Programmable resources:}

\begin{itemize}
	\item Xilinx Kintex UltraScale – XCKU040-1FFVA-1156C    FPGA 20 I/O, 530K
	Logic Cells
	\begin{itemize}
		\item speed grade: -1
		\item 20 GTH transceivers (Max Preformance 16.3 Gb/s)
	\end{itemize}
	\item MMC: LPC17762984
\end{itemize}	

\textbf{Memory:}

\begin{itemize}
	\item 512Mb  DDR3 SDRAM (32-bit interface), 800MHz (clock)
	\item 1Gb  DDR3 SDRAM (64-bit interface), 800MHz (clock)
	\item SPI Flash for FPGA configuration. Accessible by MMC
	\item SPI Flash for user data storage
	\item EEPROM with MAC and unique ID 
	
\end{itemize}

\textbf{Connectivity:}

\begin{itemize}
	\item 1 high pin count (HPC) FMC slot for single width mezzanine card
	\item Micro-USB UART connected to FPGA or MMC
	\item Stand-alone 12V power connector 
	\item MGT (Multi-Gigabit Transceiver) connected to:
	\begin{itemize}
		%		\item FMC x1
		\item RTM x16
		\item Fat\_Pipe1 x2
		%		\item AMC P2P x4
		%		\item Port 0 – possibility connected to SATA
		\item SFP x2
	\end{itemize}
	%	\item RTM connector with 8 GTP routed to it. Compatible with Sayma RTM module.
	\item  Port 0 – possibility connected to SATA
	\item RTM connector compatible with Sayma RTM module
	
\end{itemize}

\textbf{Supply:}

\begin{itemize}
	\item Monitoring of voltage and Power supply for RTM 12V and FMC 12V
	\item FMC VADJ fixed to 1V8
	\item Monitoring current of all FMC buses
	\item Stand-alone power connectore
	%	\item Czy Exar monitoruje powera? 
	
\end{itemize}


\textbf{Clocking:}

\begin{itemize}
	\item Clock recovery Si5324  is a precision clock multiplier andjitter attenuator
	\item UFL CLK input
	\item SMA CLK output
	
	
\end{itemize}


\textbf{Other:}

\begin{itemize}
	\item Temperature, voltage and current monitoring for critical power buses
	\item Temperature monitoring: FMC1, supply, FPGA core, DDR memory
	\item JTAG multiplexer (SCANSTA) for FMC access, local JTAG port and remote debug/Chipscope via Ethernet
	
\end{itemize}

\subsection{Key RTM Components}\label{key-rtm-components}
\todo[inline]{not sure if it should be here? only in RTM doc?}
\begin{itemize}

\item
  \textbf{DAC}: AD9154 4-channel high-speed data converter

  \begin{itemize}

  \item
    data rate is 1.2 GS/s at 16-bit
  \item
    clock is up to 2.4 GHz (1x, 2x, 4x and 8x interpolating modes)
  \item
    supports mix-mode to emphasize power in 3rd Nyquist Zone
  \item
    interface is 8-lane JESD204B (subclass 1)
  \item
    power consumption is 2.11 W
  \item
    each Sayma has 2 AD9154
  \end{itemize}
\item
  \textbf{ADC}: AD9656 is a 4-channel high-speed digitizer

  \begin{itemize}

  \item
    data rate is 125 MS/s at 16-bit
  \item
    clock is up to 125 MHz
  \item
    650 MHz analog bandwidth
  \item
    interface is 8-lane, 8 Gb/s per lane, JESD204B (subclass 1)
  \item
    each Sayma has 2 AD9656
  \end{itemize}
\item
  \textbf{clock generation}: (summarized \href{SinaraClocking}{here})

  \begin{itemize}

  \item
    Sayma has several distinct clock domains

    \begin{itemize}

    \item
      DAC, JESB204B output clock
    \item
      ADC, JESD204B input clock
    \item
      LO for analog mezzanines
    \end{itemize}
  \item
    These clocks may be generated using a low phase noise
    \href{ClockMezzanines}{Clock Mezzanine} PCB. A single Clock
    Mezzanine can be shared by several Sayma in a uTCA crate using
    {[}Baikal{]} PCB and an RTM RF backplane. Alternately, each Sayma
    can have its own distinct Clock Mezzanine (local generation).
  \end{itemize}
\item
  \textbf{clock distribution}

  \begin{itemize}

  \item
    HMC7043 SPI 14-Output Fanout Buffer for JESD204B
  \item
    HMC830 SPI fractional-N PLL
  \end{itemize}
\item
  \textbf{calibration ADC}: AD7194BCPZ is a 20-bit ADC for
  monitoring/calibration
\end{itemize}




